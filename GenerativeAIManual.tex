\documentclass[9pt, technote]{IEEEtran}
\IEEEoverridecommandlockouts

\usepackage{cite}
\usepackage{amsmath,amssymb,amsfonts}
\usepackage{algorithmic}
\usepackage{graphicx}
\usepackage{textcomp}
\usepackage{xcolor}
\usepackage{amsmath}
\usepackage{url}
\def\BibTeX{{\rm B\kern-.05em{\sc i\kern-.025em b}\kern-.08em
    T\kern-.1667em\lower.7ex\hbox{E}\kern-.125emX}}
\begin{document}

\title{Code Management and Generative AI Best Practices\\
    \large A Collaborative Review\\
    \large EE450 Military Robotic Applications

}

% When you contribute to this document, add an \author section for yourself
\author{\IEEEauthorblockN{Cadet 1}\\
\IEEEauthorblockA{\textit{United States Military Academy}\\
West Point, NY \\
First.Last@westpoint.edu}\\
\and
\IEEEauthorblockN{Cadet 2}\\
\IEEEauthorblockA{\textit{United States Military Academy}\\
West Point, NY \\
First.Last@westpoint.edu}
}
\IEEEauthorblockN{Cadet Taylor Brown}\\
\IEEEauthorblockA{\textit{United States Military Academy}\\
West Point, NY \\
Taylor.Brown@westpoint.edu}

\maketitle

\begin{abstract}
This is a collection of best-practice proposals by the cadets of EE450 Military Robotic Applications.
The cadets use this document to share their perspectives, experiences, and recommendations on the use
of software IDEs (specifically, Visual Studio Code and the Arduino IDE), robotics project code structure,
and the use of generative artificial intelligence in completing robotics projects. This is a
"by-cadets-for-cadets" living record that both encourages reflection and guides future cadets in their
robotics endeavors.
\end{abstract}
\section{Instructions to Contributors}
This document is divided into four parts:
\begin{enumerate}
    \item Getting Started with Arduino Programming
    \item Code Management and Structure
    \item Integrating Generative AI
    \item Testimonials and Examples
\end{enumerate}

You may contribute to any or all of these sections as much or as little as you like. It is critical that
your contributions remain \textit{high quality} and \textit{easy to understand}. Do your best to place
your inputs in the correct sections and subsections, but feel free to create your own sections or subsections
if you think you need to. You must not, under any circumstances, provide direct answers to any of the course
projects, mini-projects, quizzes, or other assignments --- the purpose is to guide other cadets on their
own problem-solving, not to solve the problem for them.

Contributions to this document must be made via a \textit{pull request} to the appropriate GitHub repository.
This first requires you to \textit{clone} the project repository and create a new \textit{branch}. If you need
guidance on how GitHub works, or how to submit a pull request, you can check the provided references.\cite{github_branch} \cite{github_pull}
\section{Getting Started with Arduino Programming}
% Use this section to discuss how to set up the IDE, how to install libraries, and how to upload code to the Arduino microcontroller.
% Any generic guidance not specific to one IDE can go here, before the subsections.
\subsection{Visual Studio Code}

\subsection{Arduino IDE}

\subsubsection{Installing Libraries}
Libraries are a crucial component and starting point towards success in EE450. There are multiple ways to download libraries, which may be necessary to troubleshoot if you run into issues, however this will describe the most common two.

\paragraph{Library Manager}
\begin{enumerate}
    \item The Library Manager in Arduino IDE looks like stacked books on the left sidebar. Other ways to navigate here are to click \texttt{Tools} and then \texttt{Manage Libraries} or click on \texttt{Sketch}, \texttt{Include Library}, and then \texttt{Manage Libraries} to navigate to the Library Manager window.
    \item Within the Library Manager, you will see a search box. Typing in this search box will search all installed and available libraries within Arduino IDE.
    \item Search for the library you need like Pixy2, Servo.h, etc. Once selected click ``Install,'' allow the system to process and show that the library is installed.
    \begin{figure}[h!]
\centering
\begin{minipage}{0.32\textwidth}
    \centering
    \includegraphics[width=\linewidth]{https://raw.githubusercontent.com/taylorbrown917-commits/EE450_Best_Practices/7e23eb898bca26e58225763291beca5383b34c06/Screenshot%202025-12-02%20143227.png}
\end{minipage}
\hfill
\begin{minipage}{0.32\textwidth}
    \centering
    \includegraphics[width=\linewidth]{https://raw.githubusercontent.com/taylorbrown917-commits/EE450_Best_Practices/7e23eb898bca26e58225763291beca5383b34c06/Screenshot%202025-12-06%20202003.png}
\end{minipage}
\hfill
\begin{minipage}{0.32\textwidth}
    \centering
    \includegraphics[width=\linewidth]{https://raw.githubusercontent.com/taylorbrown917-commits/EE450_Best_Practices/7e23eb898bca26e58225763291beca5383b34c06/Screenshot%202025-12-06%20202012.png}
\end{minipage}

\caption{EE450 Best Practices Images}
\end{figure}
\end{enumerate}

\paragraph{\texttt{.zip} Library}
Sometimes during the class you will come across a library that is not included in Arduino IDE and will require you to source the \texttt{.zip} file. Here are the easiest steps:
\begin{enumerate}
    \item Download the non-extracted version of the \texttt{.zip} file.
    \item In Arduino IDE go to \texttt{Sketch}, then \texttt{Include Library}, and add the \texttt{.zip} Library.
    \item Select the \texttt{.zip} library and open it. Wait for the program to confirm the library has been installed.
    \begin{figure}[h!]
\centering
\includegraphics[width=0.45\textwidth]{https://raw.githubusercontent.com/taylorbrown917-commits/EE450_Best_Practices/7e23eb898bca26e58225763291beca5383b34c06/Screenshot%202025-12-06%20202052.png}
\caption{Additional EE450 Best Practices Image}
\end{figure}
\end{enumerate}

\textbf{Note:} Some libraries might need some troubleshooting. Extracting all files from the \texttt{.zip} and saving them in a folder on your computer, and then moving the library into the Arduino \texttt{libraries} folder on your computer, is a way to resolve this issue and try again.


\section{Code Management and Structure}
% Below are a few recommended subsections for you to contribute to, but feel free to create your own if you think it is appropriate.
\subsection{Incorporating the Sense-Decide-Act Paradigm} 
% Did you structure your programs this way? Was it successful or not?

\subsection{Implementing States and Using State Diagrams} 
% Are state diagrams helpful at the start of a project? How did you use them?
State diagrams are extremely helpful at the start of EE450 projects even if they are not required/graded because they give you a clear visual of how your robot should behave. When you breakdown your system into states, it becomes much easier to understand what the robot is supposed to do at any given time and what events cause it to switch behaviors.

In EE450, your robot will almost always need to move between different modes (for example: \texttt{SAFE}, \texttt{FOLLOW}, \texttt{OBJECT DETECTED}, or \texttt{ATTACK/DEFEND}). A state diagram lets you map these different mdoes out before you begin to write any code.

For this project with these states, we used a state diagram to plan the overall flow of the robot. Before touching the Arduino code, we mapped out:
\begin{itemize}
    \item what each state is supposed to do,
    \item what inputs trigger a transition (like a button press or sensor reading),
    \item and what the robot should do after switching states.
\end{itemize}

Having the state diagram made implementation much easier because the code could then be built one state at a time. Each state became its own section of the \texttt{loop()} function or its own block. If something was not functioning properly, it was easier to trace which state caused the issue. Especially when working with a new platform like the Traxxas, differentiating between hardware and software issues was eased by using a state machine diagram. Also, if it is demo day and you do not have a fully working robot, having individual states from your state diagram functioning can still earn you points. Breaking your code into clear states allows you to get portions of the robot working even if the whole system is not complete. This means you can demonstrate specific state behaviors and still receive partial credit instead of ending up with no points if the full integration is giving you trouble.

Overall, state diagrams can help you:
\begin{itemize}
    \item understand the robot's behavior before coding or asking chatgpt to help you build your code,
    \item write cleaner, more concise, and organized code,
    \item help you understand chatgpt generated code more efficiently, 
    \item and troubleshoot faster when things don’t work.
\end{itemize}

They are a simple but an extremely useful tool, especially when your robot starts getting more complex. For some, in the earlier projects you may feel like the state diagram is unnecessary, however highly reccomend getting some repetitions early on, because they are extremely helpful in later projects. Others may beenfit from a state diagram from the start of Project 2 to the conclusion of Project 5. 

\subsection{File Structure and Version Management --- Avoid Drowning in Code} 
% Arduino sketches can grow into large, complex files. How did you organize them internally so you are always confident
% that your program is doing what you want?

\section{Integrating Generative AI}

\subsection{Registering for Copilot Pro} 
% provide instructions for registering for Copilot Pro using a student license

\subsection{Managing and Integrating Generated Code} 
% give guidance on how to prompt Copilot -- how do you structure the prompt, what context do you give it, and what do you
% do with what it generates? 
Using generated code from sources like ChatGPT, Copilot, etc., can be an extremely useful tool in working as a teammate to assist in fulfilling the project requirements and aiding in understanding the prompt. However, actually getting your robot to fully perform the task requirements will require you to understand the code. It is best to use AI to accelerate the progress, but not replace your understanding of solving the problem. Here are some recommended tips for prompting Copilot and other artificial intelligence platforms and managing the outputs:

\begin{enumerate}
    \item Copilot is most effective when it knows exactly what you would like it to help you with. Think of components like: What does this code do? What variables are being considered as inputs, and what are the outputs? What are the specifications and constraints of the project, including what hardware is being used in the specific project? How should the states transition?  
    Here is an example of a prompt:
    \begin{quote}
    ``Build code for the DRIVE state that reads ultrasonic sensor distance and drives the servos forward only when distance is less than 30. Print sensor values to the serial monitor.''
    \end{quote}
    \item Give Copilot the current state of your project. This means: What components have you already constructed in the process? By giving Copilot the pin mapping, defined variables, and completed state diagram, the code it generates will be easier to integrate into your own code.
    \item What do you do with what it generates? Make sure you read it carefully and ensure there are no functions or variables that conflict with your existing code. Change some components of it, like names or parameters, so that it fits with your intended design. Test the code incrementally or by itself first, this means either test the code in a new sketch or add serial prints throughout the code to ensure the behavior is congruent with your intended design. Recognize that the produced code will not be perfect, but it will help show you different methods or structures of completing tasks.
\end{enumerate}



\subsection{Documentation and Citation of Generated Code} 
% How do you document and cite generated code? How do you avoid plagiarism and ensure all generated code is clearly marked?

\section{Testimonials and Examples}
Feel free to add any other guidance, examples (good or bad!), or advice to other cadets here.

\subsection{Advice from CDT X} % example
If I had to give one piece of advice, it's to research the sensors I'm using and understand how they work, \textit{then}
build my state diagram and think about what I actually want the robot to do in terms of sensors and actuators, \textit{then}
think about how I want to prompt the generative AI. Starting by inputting the problem statement into Copilot never worked
for me a single time --- it always gave me something I didn't understand and had no idea how to fix, causing problems later.

\bibliographystyle{plain}
\bibliography{References}


\end{document}
